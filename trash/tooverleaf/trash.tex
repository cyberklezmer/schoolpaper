Our goal is to examine whether opening of schools contributes to the
epidemic and whether masks play role in schools. To this end we assume
that the expected weekly number number of newly infected children
of the $i$-th cohort outside schools is Compound Poisson with mean
\[
\beta_{t}^{i}X_{t-1},\qquad X_{t-1}=\sum_{j=1}^{5}X_{t-1}^{j},\qquad\beta_{t}=R_{t}\beta,
\]
where $\beta$ is an unknown constant and $R_{t}$ the reproduction
number.

Further, we hypothesize that the number of children of the $i$-th
cohort, infected at school, is Compound Poisson with mean 

\[
\gamma_{t}^{i}(1-M_{t-1}^{i}\mu)S_{t-1}^{i}X_{t-1}^{i}\qquad\gamma_{t}^{i}=r_{t}\gamma^{i}
\]
where $\mu^{i}$ is an unknown efficiency of masks, $\gamma^{i}$
is an unknown constant and $r_{t}=R_{t}/C_{t-2}$ is the the basic
reproduction number, 

The logic behind our model is following: without schools, the cohort
of interest is infected ``as usual'' which means that the rate of
their infection is proportional to the reproduction number and the
number of infected in the whole population (a proxy for which up to
constant is the daily number of infected). In schools, the situation
is similar with the difference that we explicitly model the contact
intensity therein (by variables $S$ and $M$ and constants $\gamma$
and $\delta$), so the reproduction number has to be stripped of the
influence of the overall contact restriction; therefore, we assume
the infection to be proportional to $r$ rather than to $R$.

Summed up, we have 

\begin{multline*}
X_{t}^{i}=\beta_{t}^{i}X_{t-1}+\gamma_{t}^{i}(1-M_{t-1}^{i}\mu^{i})S_{t-1}^{i}X_{t-1}^{i}+\epsilon_{t}^{i}\\
=\beta^{i}R_{t}X_{t-1}+\gamma^{i}r_{t}S_{t-1}^{i}X_{t-1}^{i}+\delta^{i}r_{t}M_{t-1}^{i}S_{t-1}^{i}X_{t-1}^{i}+\epsilon_{t}^{i},\qquad\delta^{i}=\gamma^{i}\mu^{i}.
\end{multline*}
where $\mathbb{E}\epsilon_{t}^{i}=0$ and $\mathrm{var}(\epsilon_{t}^{i})$
is, up to a constant, equal to the r.h.s. minus the residuum.

Finally, we assume that the observed numbers $Y_{t}^{1},\dots,Y_{t}^{5}$
of the infected are proportional to the actual ones, namely $Y_{t}^{i}=cX_{t}^{i}+e_{t}^{i}$
where $c$ is an unknown constant and $e_{t}^{t}$ are centered with
$\mathrm{var(e_{t}^{i})\sim}X_{t}^{i}$. This gives

\begin{multline*}
Y_{t}^{i}=\beta^{i}I_{t}+\gamma^{i}U_{t}^{i}+\delta^{i}V_{t}^{i}+\eta_{t}^{i},\qquad I_{t}=R_{t}Y_{t-1},\qquad U_{t}^{i}=r_{t}S_{t-1}^{i}Y_{t-1}^{i},\qquad V_{t}^{i}=r_{t}M_{t-1}^{i}S_{t-1}^{i}Y_{t-1}^{i}\\
\eta_{t}^{i}=\frac{\epsilon_{t}^{i}}{c}+r_{t}\left[\beta^{i}\sum_{j=1}^{k}e_{t}^{j}+(\gamma^{i}S_{t-1}^{i}+\delta^{i}M_{t-1}^{i}S_{t-1}^{i})e_{t}^{i}\right]
\end{multline*}
Note that $\mathbb{\ensuremath{E}}\eta_{t}^{i}=0$ and $\mathrm{var(}\eta_{i}^{i})=\sum_{j}d_{t}^{i,j}X_{t}^{j}$
for some $d_{i,t}^{t}$. As, in practice, the ratio of $X_{t}^{i}$
does not vary much in time, we approximate $\text{var}(\eta_{i}^{i})\doteq dX_{t}\doteq\frac{d}{c}I_{t}$
where $d$ is unknown parameter. 

Our hypotheses are 

\[
H_{0}^{i}:\gamma^{i}=0\text{ against }H_{1}^{i}:\gamma^{i}>0
\]
(schools do not/do have influence) and 
\[
\tilde{H}_{0}^{i}:\delta^{i}=0\text{ against }\tilde{H}_{1}^{i}:\delta^{i}<0
\]
(masks do not / do have influence).

We use WLS to estimate coefficients in (\ref{eq:wls}), i.e. OLS after
dividing the equation by $\sqrt{I_{t}}$ for each $t$, and test the
hypotheses by $t$-tests for each $i.$
\end{document}

Further, assuming that, only a ratio $c$ of cases is reported, we
may divide (\ref{eq:x}) by $c$ to get 
\begin{equation}
Y_{t}\doteq r_{t}C_{t-1}Y_{t-1},\qquad Y_{t}^{i}\doteq\alpha^{i}r_{t}C_{t-1}Y_{t-1}+\gamma^{i}r_{t}S_{t-1}^{i}Y_{t-1}^{i},\qquad1\leq i\leq4,\label{eq:y}
\end{equation}
where $Y_{t}\doteq cX_{t}$ is the overall reported number of infections
and $Y_{t}^{i}\doteq cX_{t}^{i}$ is the reported infections number
in the $i$-th cohort. By dividing by the size of the cohort $i$,
we get
\begin{equation}
P_{t}^{i}\doteq\beta^{i}r_{t}C_{t-1}P_{t-1}+\gamma^{i}r_{t}S_{t-1}^{i}P_{t-1}^{i},\qquad P_{t}^{i}=\frac{Y_{t}^{i}}{s^{i}},\qquad P_{t}=\frac{Y_{t}}{s},\qquad\beta^{i}=\frac{s}{s^{i}},\label{eq:prp}
\end{equation}
where $s_{i}$ is the size of the $i$-th cohort and $s$ is the whole
population size. 

Clearly, adding cohort $X^5_{i,t}=\sum_{j} Z^j_{i,t} - \sum_{j} X^j_{i,t,}$ and summing over $1\leq j \leq 5$ gives
\begin{equation}
X_{i,t} = (\alpha+\beta) Q_t + e_{i,t}
\label{eq:xi}
\end{equation}
which, by summing over all $i$, gives (\ref{eq:x}) provided that $\alpha + \beta = 1$.


\begin{equation}
X_{i,t} = f_t(d_i Y_{t-1}) + \beta_i Q_{i,t}
+ \gamma U_{i,t} + \delta V_{i,t} + e_{i,t}
\label{eq:x}
\end{equation}
$$
Q_{i,t}=D_tC_{t-1}Y_{i,t-1} , \qquad U_{i,t}=D_t N_t X_{i,t-1}
\qquad V_{i,t}=D_t M_t X_{i,t-1}
$$

 $d_i$ is the fraction of population living in the $i$-th district
 
 
 The values of contact reduction $C_t$ were taken from the longitudinal sociological survey \cite{paqcovid}. The trend $r_{t}$ is computed by double exponential smoothing of series
$\frac{Y_{t}/Y_{t-1}}{C_{t-1}}$ with parameter $\alpha=0.2,\gamma=0.5$.  


The values of $N$ and $M$ are estimated from publicly available
sources, mostly from resolutions of the Czech government concerning
school attendance, which have been put into effect through decrees
of the Ministry of Education (see the electronic supplement for
details).